\chapter{Прикладное использование}
\section{Работа с интервалами} % 5.1 Handling Intervals
В определении CBRP доказывающий должен показать, что зафиксированное целое положительное число превышает определенное пороговое значение, в то время как классические доказательства диапазона предназначены для того, чтобы показать, что зафиксированное целое положительное число находится в пределах интервала, то есть меньше верхнего порогового значения и больше нижнего порогового значения.

Примитивный способ расширения основной конструкции для обработки доказательств диапазона интервалов заключается в создании двух отдельных обязательств для значения $x$: одно для доказательства диапазона $x \geq t$ для некоторого порогового значения $t$, и другое для другого доказательства диапазона $N - x \geq N - t$. Посколько оба обязательства представляют собой хеш-результаты, их можно объединить с помощью другой операции хеширования для получения корневого обязательства, которое проверяющий может использовать для проверки обоих доказательств диапазона.

Стоит помнить, что основная конструкция критически полагается на корректность сформированного обязательства. В противном случае у доказывающего была бы возможность генерировать доказательства диапазона, которые кажутся действительными для проверяющего, но не соответствуют корректно сформированному обязательству.

Обобщив, можно показать, что обязательства основной конструкции конъюнктивны. Это позволяет применять их для объединений нескольких доказательств по набору секретных целых положительных чисел и их обязательств вместе, а не только для одного целого положительного числа. В терминах приложений это может быть использовано для:
\begin{itemize}
	\item доказательств включения двумерной точки в заданную область поверхности;
	\item финансовых транзакций в системах, где необходимо проверить, что пользователь владеет определенным количеством активов в разных валютах
	\item биометрических данных в системах, где значения имеют числовые эквиваленты.
\end{itemize}

% TODO: картинка?

\section{Повторное использование доказательств} % 5.2 Reusable Proofs
Одним из ограничений основной конструкции является тот факт, что нулевое разглашение не гарантируется, если необходимо сгенерировать несколько диапазонных доказательств для одного и того же обязательства.
Чтобы разрешить повторное использование без ущерба безопасности, можно интерпретировать обязательство как лист сбалансированного двочиного дерева Меркла, крень которого будет действовать как повторно используемое обязательство для проткола.
С $T$ листьями в этом дереве Меркла новый протокол сможет поддерживать $T$ различных диапазонных доказательств по одному и тому же обязательству, при этом увеличивая размер доказательства и время проверки только на $\O(\log T)$.
Однако время генерации обязательства и доказательства увеличивается в $\O(T)$.
Основываясь на результатах, представленных в Таблице \ref{table:2}, 60-кратное повторное использование обязательства по основанию $16$ на базе хеш-функции Blake3 все равно будет быстрее, чем генерация доказательства Bulletproofs.
Данная техника аналогична тому, что используется в схеме подписи Меркла \cite{10.1007/0-387-34805-0_21}.

\section{Предотвращение изменяемости}
Важным аспектом системы хеш-цепей является тот факт, что доказательства являются изменяемыми: доказательства значения $311 < 312$ может быть повторно использовано для доказательствао того, что $300 < 312$ или $211 < 312$, так как $311$ доминирует над данными значениями.
Данный аспект может не всегда желателен.

Например, под изменяемостью можно понимать возможность повторного использования одного и того же доказательства для различных запрашиваемых диапазонов, без необходимости повторго вычисления каждого их них с нуля.
В качестве примера, полные копии платежных ведомостей являются распрастраненным запросом и служат важным показателем для арендодателей, посколько показывают, что потенциальный арендатор имеет стабильных доход и сможет производить ежемесячные платежи.
В целях конфиденциальности арендатор могут запросить обязательство у работодателя, а затем вычислить одно единственное доказательство диапазона для наивысшего запрашиваемого значения и, если возможно, повторно использовать его для доказательства меньших диапазонов другим арендодателям.
Аналогично, агентам по недвижимости может потребоваться одно единственное доказательство максимального диапазона от арендаторов, затем брокеры смогут перенастроить эти доказательства для кажждого арендодателя без взаимодействия арендатора и арендодателя.

В других случаях это считается атакой с адаптивным выбором сообщений, когда подделанное доказательство диапазона подтверждает диапазон, отличный от значения, которое было легитимно сгенерированно.
В таких случаях изменяемости следует избежать.
Например, оригинальный протокол PayWord по умолчанию является изменяемым, и при использлвании для ставок на аукционах злоумышленник может изменить доказательства диапазона и подайть действительную ставку, меньшую оригинальной.
Изменение ставок приведет к манипуляциям на аукционе в пользу недобросовестных пользователей.

Существует как минимум два подхода для защиты от изменяемости.
Прямолинейное решение заключается в том, что доказывающий всегда должен подписывать доказательство и запрашиваоемое числовое значение диаппазона.

Тем не менее, второй подход позволяет сделать основную конструкцию автоматически защищенной от изменяемости без необходимости в дополнительных подписях, используя идеи из основанных на хешах постквантовых схем.
На самом деле, можно легко найти сходства между основной конструкцией и схемой подписи Winternitz (WOTZ) \cite{cryptoeprint:2011/191, 10.1007/0-387-34805-0_21}.
Часть хеш-цепи можно рассматривать как вариацию WOTS, с ограничениями для доверенного лица на значения, которые могут быть подписаны или доказаны.
В то время как в WOTS любая комбинация битов может быть подписана, в основной конструкции доверенное лицо гарантирует, что хеш-цепи вычисляются таким образом, что не все значения имеют путь к корню дкоазательства.
WOTS защищает от атак с адаптивным выбором сообщений, подписывая контрольное значение, добавленое с байтам сообщения, за счет дополнительных хеш-цепей.
Основную констркуцию можно улучшить с помощью точно такой же методики.

Каждая хеш-цепь вычисляется только в одном направлении из-за вычислительной невозможности найти праобраз хеша. Контрольная сумма вычисляется как $C = \sum^n_{i = 1} (b - 1 - M_i)$, где $M_i$ - $i$-тый элемент разложения по основанию $b$.
Аналогично WOTS, полученная контрольная сумма кодируется как цело число по основанию $b$ и добавляется к доказатульству.
Например, в $32$-битной области, число $2^{32} - 1$ кодируется как $(255, 255, 255, 255)$ по основанию $256$ и будет иметь контрольную сумму равную $0$. Аналогично, сообщение $(0, 0, 0, 1)$ будет иметь контрольную сумму $1019$, которая кодируется как $(3, 251)$ по основанию $256$.
На практике, при использовании основания $256$, контрольная сумма добавляет две хеш-цепи, что приводит к увеличению длины доказательств на $64$ байта для диапазонов до $8192$ бит.

Подводя итог, подход с контрольной суммой WOTS работает потому, что если злоумышленник попытается доказать меньший диапазон, то контрольная сумма увеличиться, но из-за одностороннего характера хеш-функци, создать более длинные хеш-цепи для контрольной суммы не выйдет.

\section{Применения} % 5.4 Applications
Данный раздел описывает набор прикладных областей для применения основной конструкции в различных практических условиях.
Особенно стоит отметить применение в ограниченных средах, таких как использование в смарт-картах или датчиков интернета вещей (Internet of Things - IoT), или когда алгебраические групповые примитивы либо недоступны, либо слишком сложны для безопасной реализации, например в аппаратном обеспечении, встроенном программном обеспечении или языках программирования, где нет существующей библиотеки для Bulletproofs и других систем доказательст с нулевым разглашением.
Однако следует отметить, что обязательства основной конструкции не являются гомоморфными и, следовательно, не могут использоваться для сложения или вычитания конфиденциальных сумм.

CBRP могут бять расширены для доказательства того, что многомерные данные достоверно находятся в пределах заданного диапазона, например двухмерное или трехмерное местоположение, путем выдачи доказательства для комбинированной долготы и широты или координат по трем осям $x$, $y$ и $z$.
Существует множество зависящих от местоположения приложений, включая отслеживание мобильных и веб-пользователей с учетом конфиденциальности и ограничения по местоположению или области для приемников спутникового телевидения и радиостанций.
Особенно многие приложения в области облачных вычислений завичят от осведомленности о местоположении, включая испольнительные устройства и беспроводные датчики IoT \cite{YANG2018799}. Существуют различные криптографические примитивы, безопасность которых зависит от проверяемого местоположения, включая приватное определение близости местоположения \cite{Narayanan2011LocationPV}, обмен ключами на основе позиции и многопартийные вычсления \cite{10.1007/978-3-642-03356-8_23}, ключи идентификации на основе местоположения \cite{Chalkias2010SecureCP} и контроль доступа \cite{10.1007/978-3-642-22497-3_16}.
Одним из важных аспектов является наличие злоумышленников, предоставляющих поддельные данные о местоположении, и ненадежные координаты, сообщаемые пользователями.
Для частичного решения данной проблемы, была введена концпеция проверки местоположения, которая сосредоточена на методах безопасной проверки предоставленных данных о местоположении, изначально предложенная в \cite{10.1007/3-540-48285-7_30} и далее иследованная в \cite{e76f5acad5d74705bd527e4b44d1aa9a, 10.1007/978-3-642-03356-8_23, 10.1145/941311.941313, 1542879, 4012649, YANG2018799, 1618808}.
Обазательства CBRP могут быть выданы датчикам позиционирования IoT, например подписаны встроенной защищенной средой выполнения, и затем приложение рещает, какое доказательство диапазона сгенерировать.

Широкая область применения, основанная на данных конкретных примерах, включает в себя процедуру проверки клиентов Know Your Customer - KYC.
Регуляторы обязали финансовые учреждения и сети внедрять валидацию KYC, которая может использоваться для изменения местоположения клиента и обеспечения конфиденциальности возраста с помощью CBRP.
Данные технологии примененимы и к другим аспектам финансовых услуг, где они могут использоваться для представления баланса счета, депозитов и доходов, которые могут быть подтверждены доверенными третьими сторонами, такими как банки.
В таком случае, данная технология примененима для заявок на ипотеку, кридитные карты и кредиты, так как право собственности на активы может быть доказано проверяющему без раскрытия личной информации.
Как уже упоминалось, можно предоставить доказательство арендодателю при подаче заявки на аренду, не раскрывая информации о личном состоянии или доходах.

Изначально PayWord - это кредитный микроплатежный протокол, который включает три стороны: клиента, продавца и брокера, и использует хеш-цепи для повышения вычислительной производительности.
При выдаче банковских чеков, где тратить средства можно постепенно, пока полная сумма не будет исчерапа, продавцы не должны обязательно знать об оставшемся балансе покупателя.
Данный формат идеально вписывается в блокчейн-приложения цифровых переводов, где владельцы счетов выдают чеки вне основного блокчейна.
Однако, хотя системы микроплатежей сосредоточены на малых значениях, сегодняшние малые номиналы во многих приктовалютах и экстремальная волатильность обменных курсов могут привести к длинным хеш-цепям в оригинальной схеме PayWord, даже при выдаче чеков на несколько единиц валюты.
Посколько основная конструкция эффективна для больших диапазонов и многие блокченый позволяют использовать смарт-контракты с поддержкой Тьюринг-полных инструкций, включая встроенные функции хеширования, данный протокол может стать отличным кандидатом для основанных на блокчейне постепенно выкупаемых чеков.

Аналогично были предложены варианты PayWord для использования в аукционах с сохранением конфиденциальности, где, например, победитель может не раскрывать публично свою финальную ставку, а лишь доказать, что ее значение больше или меньше, в зависимости от правил аукциона, остальных ставок.
В аукционах с запечатанными ставками Викри \cite{https://doi.org/10.1111/j.1540-6261.1961.tb02789.x}, например, диапазон выигрышной ставки может оставаться секретом для всех, так как окончательная цена, которую платят, является второй по величине ставкой.
Основная конструкция также может быть кандидатом на протокол в данных условиях.

Помимо финансовых услуг, CBRP также могут быть применены для достижения конфиденциальности временных меток для сертификатов, выданных удостоверяющим центром.
С помощью CBRP организация, получившая сертификат от удостоверяющего центра, может доказать третьей стороне, что срок действия сертификата на данный момент не истек, не раскрывая дату истечения.
Другие метаданные, такие как время выдачи сертификата или IP-адрес, также могут быть включены в доказательство диапазона.
CBRP делают сертификаты более защищенными с точки зрения конфиденциальности для клиентов, не отказываясь при этом от серверной проверки.

Другие потенциальные случаи использования могут включать доказательства диапазонов рангов в резюме или онлайн-таблицах прохождения в учебные заведения без раскрытия фактической позиции в рейтинге.
В некоторых случаях это может быть желательно, когда требование к работе  или зачисление в университет предполагает, что необходимо лишь превысить определенный порог.
Аналогично, производители продуктов питания и лекарств, которым требуется соответствовать определенным верхним или нижним границам значения ингридиентов для получения лицензии, могут получить подписанны обязательства CBRP от управляющего органа в сфере продуктов питания, лекарственных препаратов или аналогичных учреждений, а затем предоставить проверяемое доказательство того, что законные пределы соблюдены, не раскрывая полные рецепты своей продукции.
