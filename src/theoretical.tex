\chapter{Теоретическая часть}
\section{Введение в основные понятия} % 2 Preliminaries
В данной работе, $\lambda$ обозначает параметр безопасности.
Для целого положительного числа $N$, $[N]$ обозначает набор целых положительных чисел $\{1, 2, \dots, N\}$.
Для двух целых положительных чисел $a$ и $b$, таких что, $a < b$, запись $[a, b]$ обозначает набор целых положительных чисел $\{a, a + 1, \dots, b\}$.
Будем называть алгоритм эффективным, если по отношению к $\lambda$ он работает за полиномиальное время.
Величину, обратно пропорциональную полиномиальному значению по отношению к $\lambda$, назовем незначительной.
% TODO: Проверить \approx
Для двух распределений $D_1$ и $D_2$ по одной и той де области будем писать $D_1 \approx D_2$, чтобы обозначить вычислительную неразличимость; а именно, для любого сколь угодно эффективного противника, который получает на вход образец и выводит по нему бит, разница в вероятности того, что противник выведит $1$ при получении образца из $D_1$, и вероятности того, что противник выведит $1$ при получении образца из $D_2$, незначительна.
Вышеуказанную величину назовем преимуществом противника в различении $D_1$ и $D_2$.

% TODO: Проверить <>
Обозначим битовую строку произвольной длины как $\{0, 1\}^*$.
Операцию конкатенации строк обозначим как $||$.
Для целого положительного числа $n$, обозначим упорядоченную последовательность набора элементов $\{a_1, \dots, a_n\}$ как $<a_1, \dots, a_n>$.

Запись $y_{i, 1}, \dots, y_{i, d}$ обозначает представление целого положительного $d$-значного числа $y_i$ по основанию $b$.

\subsection{Доказательство диапазона на основе удостоверений} % 2.1 Credential-Based Range Proofs
% TODO: Проверить \C, |P, \Pi. Добавить \text?
Для фиксированного целого положительного числа $N$, пространства обязательств $\C$, пространства доказательст $\P$ и пространства случайнойсти $\{0, 1\}^\lambda$, неинтерактивный CBRP представляет собо набор алгоритмов $\Pi = (Setup, Commit, Prove, Verify)$ со следующими свойствами:
- $Setup(1^\lambda) \rightarrow pp$. Алгоритм установки на основе параметра безопасности $\lambda$ генерирует публичные параметры $pp$.
- $Commit(pp, x; r) \rightarrow com$. Алгоритм обязательства на основе публичных параметров $pp$, целого положительного числа $x \in [N]$ и случайной битовой строки $r \in \{0, 1\}^\lambda$ генерирует обязательство $\com \in \C$.
- $Prove(pp, x, t; r) \rightarrow \pi$. Алгоритм доказательства на основе публичных параметров, целого положительного числа $x$, порогового значения $t \in [N]$ и случайной битовой строки $r \in \{0, 1\}^\lambda$ генерирует доказательство $\pi$.
- $Verify(pp, com, t, \pi) \rightarrow z$. Алгоритм проверки на основе публичных параметров, обязательства $\com \in C$, порогового значения $t \in [N]$ и доказательства $\pi$ генерирует бит $z \in \{0, 1\}$.

% TODO: Security

% TODO: Definition (Навязывающее обязательство)
% TODO: Обозначение злоумышленника (\A?)
Обязательство для $CBRP$ назвыается навязывающим если для любого сколь угодно эффективного злоумышленника $A$, который получает публичные параметры $pp \leftarrow Setup(1^\lambda)$ и генерирует пару раличных целых положительных чисел $x_0, x_1 \in [N]$, пару случйаный битовых строк $r_0, r_1 \in \{0, 1\}^\lambda$ и два обязательства $com_0 \leftarrow Commit(pp, x_0; r_0)$ и $com_1 \leftarrow Commit(pp, x_1; r_1)$, величина $Pr[com_0 = com_1]$ незначительна по отношению к $\lambda$.

% TODO: Definition (Идеальная? полнота)
% TODO: Обозначение злоумышленника (\A?)
CBRP удовлетворяет (идеальной?) полноте если любой сколь угодно эффективный злоумышленник $A$, генерирующий $x, t \in [N]$, причем  $x \geq t$, и $r \in \{0, 1\}^\lambda$, будет получать $Verify(Commit(x; r), t, Prove(x, t; r)) = 1$.

% TODO: Definition (Обязательно-условная корректность)
% TODO: Обозначение злоумышленника (\A?)
CBRP удовлетворяет обязательно-условной корректности если для любого сколь угодно эффективного злоумышленника $A$, который на основе публичных параметров $pp \leftarrow Setup(1^\lambda)$ генерирует набор $(x, t, r, \pi) \in [N] \times [N] \times \{0, 1\}^\lambda \times \P$, где $x < t$, вероятность $Verify(pp, Commit(pp, x; r), t, \pi) = 1$ незначительна по отношению к $\lambda$.
