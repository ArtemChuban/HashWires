\chapter{Введение}
Доказательство диапазона - это особый тип доказательства с нулевым разглашением (ZKP - Zero Knowledge Proof) \cite{10.1145/22145.22178}, предназначенное для подтверждения того, что зафиксированное целое число находится в пределах некоторого интервала, не раскрывая при этом никакой дополнительной информации о самом числе.
Gregory Maxwell использовал доказательства подобного рода для достижения конфиденциальных транзакций \cite{maxwell2016first}, сумма которых скрыта в обязательстве Педерсена, а доказательство диапазона позволяет проверяющему убедиться, что она находится в неотрицательном интервале.
Доказательства диапазона с нулевым разглашением впервые были предложены в работах Дамгорда \cite{eurocrypt-1993-2249}, Фудзисаки и Окамото \cite{10.5555/646762.706160}, а с более практическими конструкциями у Бодута \cite{10.1007/3-540-45539-6_31}, Камениша и других \cite{10.1007/978-3-540-89255-7_15}, и, что особенно примечательно, Bulletproofs от Бюнца и других \cite{8418611}.

Типичное доказательство диапазона с нулевым разглашением (ZKRP - Zero Knowledge Range Proof) включает в себя криптографические обязательство (commitment) и доказательство (proof).
Обязательство должно удовлетворять условиям, известным как сокрытие (hiding) и навязывание (binding), а доказательство должно соответствовать требованиям полноты (completeness), корректности (soundness) и нулевого разглашения (zero-knowledge).
ZKRP использовались для создания приватных транзакций в блокчейнах (например, предложение Gregory Maxswell по обеспечению конфиденциальных транзакций \cite{maxwell2016first} и MimbleWimble \cite{poelstra2016mimblewimble}) и для смарт-контрактов на Ethereum, обеспечивающих конфиденциальность.
Также было предложение использовать их для обработки доказательств платежеспособности \cite{cryptoeprint:2020/468, 10.1145/2810103.2813674}, обеспечения конфиденциальности в системах электронных голосований и аукционов, а также для анонимных удостоверений \cite{morais2019survey}.

Некоторые практические приложения ZKRP существуют в контексте, в котором некоторое доверенное лицо выступает в качестве авторитета для создания или проверки обязательств по целым значениям.
В качестве иллюстративного примера рассмотрим ситуацию, в которой гражданин хочет доказать поставщику услуг, что его возраст превышает определенный порог, не раскрывая при этом свою дату рождения \cite{asecuritysite_11123, shah2017zeroknowledge}.
Доверенное лицо (например, государственный орган, выдающий удостоверения личности) может выдать заявление, подтверждающее возраст доказывающего, который в свою очередь имеет возможность использовать протокол ZKRP для создания доказательства, которое убедит проверяющего в том, что его возраст действительно превышает запрашиваемый порог.

Не смотря на то, что вышеописанный сценарий можно решить при помощи существующих конструкций ZKRP, таких как Bulletproofs или обязательство Педерсена, эти решения часто оказываются менее практичными в условиях ограниченных ресурсов, где асимметричные криптографические примитивы считаются дорогими.
В приложениях, где эти доказательства необходимо генерировать и проверять очень часто, более простые и легковесные примитивы могут стать привлекательной альтернативой для условий, в которых существует доверенная сторона, генерирующая обязательства.

В частности, можем ли мы использовать предположение о том, что проверяющий доказательства диапазона может доверять тому, что обязательство было сгенерированно честно, чтобы получить более простые и эффективные конструкции.
Чтобы учесть дополнительное требование доверенное стороны, введем доказательство диапазона на основе удостоверений (CBRP - Credential-Based Range Proof), которое отличается от ZKRP двумя пунктами:
- Требование корректности упрощено до более слабого его варианта - обязательно-условная корректность (commitment-conditional soundness)
- Требование нулевого разглашения упрощено до более слабого его варианта - неразличимость свидетелей (witness indistinguishably)

Чтобы интуитивно понять эти ослабления требований к безопасности для CBRP, начнем с описания простой схемы на основе хеширования для доказательств диапазона, изначальное представленной Rivest и Shamir \cite{10.1007/3-540-62494-5_6} в протоколе, известном как PayWord (с адаптациями).
Дано две устойчивые к коллизиям хеш функции $G$ и $H$, и целое число $N$, отображающее максимально возможное значение области.
Эмитент создает обязательство $c$ для некоторого секретного целого числа $k \in [0, N]$, выбирая случайное начальное значение $r$, рассчитывая $c = H^k(G(r))$, где $H^k(\cdot)$ - $k$ последовательных применений функции $H$, и размещает $c$ в публичном месте или отправляет подписанное $c$ доказывающему напрямую.
Эмитент отправляет случайное начальное значение $r$ и целое число $k$ доказывающему, который сможет создать доказательство диапазона на основе $r$ для некоторого порогового значение $t$, вычислив $\pi = H^{k - t}(G(r))$ и отправив $\pi$ проверяющему.
Проверяющий проверяет, что $H^t(\pi) = c$, чтобы убедиться, что $c$ является обязательством некоторого значения $x \geq t$.

Несмотря на свою простоту, PayWord имеет значительный недостаток: пусть $n = \lceil \log_2 N \rceil$ - количество бит, необходимых для представления размера области $N$.
Обратим вниманием, что генерация обязательства и доказательства, а также их проверка имеют асимптотически экспоненциальную сложность по отношению к $n$.
Таким образом, использование PayWord в качестве CBRP в действительности подходит только для областей малых размеров, а его производительность для больших диапазонов вряд ли будет конкурентоспособной по сравнению с общими ZKRP.

Совершенно другой подход с построению CBRP для областей больших размеров без использования ZKRP включает в себя применение закодированных схем.
Для некоторого секретного целого числа $k \in [0, N]$, эмитент может создать обязательство, представляющее собой хеш закодированное схемы $C$, которая вычисляется как $C(k, y) = 1$, если $k \geq y$ и $0$ в противном случае, вместе с обязательствами для случайных значений, каждое из которых связанно с одним из входов цепи.
Доказывающий, с некоторым входным пороговым значением $t$, генерирует доказательство, которое состоит из закодированное из закодированной схемы $C$, а также раскрывает случайные значения, относящиеся к входам сети для битового представления $t$.
Проверяющий может убедиться, что открытые обязательства соответствуют правильным битам $t$, хеш закодированных схем соответствует том, что содержится в финальном обязательстве, и вычисление закодированной схемы возвращает ожидаемый результат.
Интуитивно, обязательно-условная корректность вытекает из корректности схемы кодирования, а неразличимость свидетелей следует из конфиденциальности самой закодированной схемы.
Это решение явно обобщается на произвольные схемы, вместо того, чтобы быть ограниченным только обработкой доказательств диапазона, но размер доказательств в данной схеме довольно велик, в основном из-за того, что оно должно содержать полное описание закодированной схемы.
Таким образом, хотя этот подход технические имеет возможность обрабатывать входные данные большего диапазона, которые PayWord не мог, все же существует значительное пространство для усовершенствований.

% TODO: 1.1 Our Contributions
% TODO: 1.2 Related Work
